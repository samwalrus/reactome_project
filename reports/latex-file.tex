\documentclass[•]{article}
\begin{document}

	\title{An Application of ILP to Systems Biology}
	\author{Samuel R Neaves}
	\date{June 2015}
	\maketitle
	
	%The following line titles the abstract as blank
	\renewcommand{\abstractname}{}
	\begin{abstract}
	\textbf{Abstract.} We show how a logical aggregation method combined 
	with propositionalization methods can construct novel structured 
	biological features for 	classification and ranking tasks in 
	systems biology. 
	\end{abstract}
	\section{Introduction and Background}
	This document describes a method to use Inductive Logic Programming
	(ILP) techniques to mine structured biological data. 
	The biological problem addressed in this paper to demonstrate the
	appropriateness of ILP to the task of classifying Lung Cancer samples
	For details about the motivation of this task please see {DREAM CHALLEENGE 
	REF}
	In machine learning terms this can be formalised as a classification task
	or more generally as a ranking task. 
	Often bioinformatic research is concerned with identifying differences
	between sets of samples. It is common for this to be done using an 
	attribute value representation where samples are vectors of probe 
	measurements. Differences are then shown as lists of genes.
	This ignores known relations.
	The ILP community has often been seen as appropriate to biological 
	data before 1.GO subgroups,2 Network construction and amendment. 3 
	classifying samples.
	This work is most similar to 3. 
	In "Using Bio-Pathways in Relational Learning 2008" the authors propose
	using fully coupled fluxes. An FCF is the longest chain FCF is the 
	longest    possible chain of vertices in which non-zero vertex activation
	implies a certain (non-zero) activation in its successors.
	Researchers have used aggregation methods such as avg and svd.
	Machine Learning is applied to a task, the output of a machine learning 
	is a model. The appropriateness of the model format to the task.
	Biological researchers are interested in structured outputs with network
	motifs and consistency modelling research areas. 
	In some sense classification is not the task. 
	\subsection{GEO Data}
	
	\subsection{Biological Pathway Data}
	Reactome is a collection of manually curated peer reviewed pathways.
	Reactome is made availble as an RDF xml file. Which allows for simple 
	passing using SWI-Prologs semantic web libraries. 
	  
	\section{Method}
	The idea is to use structural information as	to enable an 
	appropriate level of abstraction for the interpretation of the 
	model to be insightful for a biologist. 
	In this work we use a high level logical abstraction of the idea of 
	a biological pathway. 
	Directed Networks of genes and entities are not appropriate, either 
	bipartite graphs or hypergraphs are required.  
	We extract a directed reaction centric graph. In this way we 
	address the problem identified in X. Boolean networks are a common
	abstraction, but these normally are at the gene or protein level
	not at the reaction level. To use a boolean network abstraction on
	a reaction network, we apply a logical aggregation method 
	  \subsection{Barcode}
	  Barcode is a tool for applying learnt thresholds to microarray data
	  This gives a binary value for each probe in the microarray. This tool 
	  is implemented as a R package.
	  \subsection{Logical Aggregation}
	  Using the Reactome RDF graphs and biological 
	  Linking Reactions
	  Removing trivial entities. 
	  \subsection{Propositionalization}
	  It has been recommend to try prop methods first when attempting ILP
	  
	     \subsubsection{Warmr}
	     Warmr is the first order equivalent of item set mining.
	     It can be used as propo method by searching for frequent 
	     queries in the two classes, and taking the set difference 
	     (not sure if this would work)
	     We can define predicates such as path or loop
	     As warmr is not prunning by classificaiton accurracy
	     it can quickly build to an intractable search.
	     
	     \subsubsection{Tree Liker}
	     Tree liker is a modern ILP tool that implements a number of algos
	     it has been shown to produce long features, which are desirable for 
	     this task.
	     \subsubsection{Combined Search}
	     
	     
	  \subsection{Ranking and Classification}
	  As we are interested in producing comprehensible models, we limit
	  our experiments to rule and tree building algorithms.
	  We use 
	     \subsubsection{Appropriate Metrics}
	     Dream challenge uses the following metrics
	     Flach et al recommend these metrics. 
	     
	\section{Results}
	ROC curve
	Features Found and models built.
	\section{Discussion}
	This work has shown the appropriateness of using ILP methods
	to mine the abundance of highly structured biological data.
	Using this method we have identified differences in pathway activation
	patterns that go beyond the standard analysis of differentially expressed
	genes, enrichment analysis, gene feature ranking and pattern mining for 
	common network motifs (by using pre-defined background knowledge 
	predicate in 	warmr such as onpath. The use of logical aggregation to a  
	reaction graph 
	simplifies the search for hypothesises to an extent where all pathways
	can be searched in reasonable time. 
	The extension of treeliker found features with warmr to connect loops is
	believed to be novel. This allows for the finding of loops with additional
	properties in a reasonable amount of time. 
\end{document} 