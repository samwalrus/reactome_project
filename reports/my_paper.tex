
%%%%%%%%%%%%%%%%%%%%%%% file typeinst.tex %%%%%%%%%%%%%%%%%%%%%%%%%
%
% This is the LaTeX source for the instructions to authors using
% the LaTeX document class 'llncs.cls' for contributions to
% the Lecture Notes in Computer Sciences series.
% http://www.springer.com/lncs       Springer Heidelberg 2006/05/04
%
% It may be used as a template for your own input - copy it
% to a new file with a new name and use it as the basis
% for your article.
%
% NB: the document class 'llncs' has its own and detailed documentation, see
% ftp://ftp.springer.de/data/pubftp/pub/tex/latex/llncs/latex2e/llncsdoc.pdf
%
%%%%%%%%%%%%%%%%%%%%%%%%%%%%%%%%%%%%%%%%%%%%%%%%%%%%%%%%%%%%%%%%%%%


\documentclass[runningheads,a4paper]{llncs}

\usepackage{amssymb}
\setcounter{tocdepth}{3}
\usepackage{graphicx}
\usepackage{cite}
\usepackage{url}
\usepackage{natbib}
\usepackage{etoolbox}
\urldef{\mailsa}\path|{samuel.neaves, sophia.tsoka}@kcl.ac.uk|


\renewcommand{\bibname}{\leftline{References}}
   
\newcommand{\keywords}[1]{\par\addvspace\baselineskip
\noindent\keywordname\enspace\ignorespaces#1}

\begin{document}

\mainmatter  % start of an individual contribution


\title{ILP Identifies Pathway Activation Patterns}
% oe Identifying Pathway Activation Patterns- For Classification
%ILP identifies pathway activation patterns. 
%n Application of ILP to Systems Biology
% a short form should be given in case it is too long for the running head
\titlerunning{ILP Identifies Pathway Activation Patterns}

% the name(s) of the author(s) follow(s) next
%
% NB: Chinese authors should write their first names(s) in front of
% their surnames. This ensures that the names appear correctly in
% the running heads and the author index.
%
\author{Samuel R Neaves, Dr Sophia Tsoka}
%
\authorrunning{ILP Identifies Pathway Activation Patterns}
% (feature abused for this document to repeat the title also on left hand pages)

% the affiliations are given next; don't give your e-mail address
% unless you accept that it will be published
\institute{Department of Informatics King's College London,\\
Strand, London, UK\\
\mailsa\\
%\mailsc\\
\url{http://www.kcl.ac.uk}}

%
% NB: a more complex sample for affiliations and the mapping to the
% corresponding authors can be found in the file "llncs.dem"
% (search for the string "\mainmatter" where a contribution starts).
% "llncs.dem" accompanies the document class "llncs.cls".
%

\toctitle{ILP Identifies Pathway Activation Patterns}
\tocauthor{Samuel R Neaves}
\maketitle


\begin{abstract}
We show a logical aggregation method that combined with propositionalization methods can construct novel structured biological features for classification and ranking tasks in Systems Biology. 

\keywords{Biological Pathways, Reactome, RDF, Barcode, Logical Aggregation, Warmr, Treeliker, ILP.}
\end{abstract}



\section{Introduction and Background}

%This paper describes a method to use Inductive Logic Programming(ILP) techniques to mine structured biological data. 
%Overview
In the field of Systems Biology researchers are often interested in identifying perturbations within a biological system that are different across experimental conditions. In this paper we use the example of identifying these perturbations in two types of Lung Cancer.
%To clarify we are looking to identify differences rather than building a diagnostic tool for Lung Cancer.

% typical approach
A typical pipeline for this kind of task has three distinct stages. The first stage is to use a technology such as a microarray or RNAseq to measure gene expression across the genome in a number of samples from each of the experimental conditions. 
The second stage is to identify a subset of genes whose expression values correlate with the conditions.
Stage 2 is commonly achieved by performing differential expression analysis and ranking genes by their fold change values or other statistic. A statistical test is then used to identify the relevant set to take forward to stage 3. 
Alternatively for stage 2 researchers may train a model using machine learning to classify samples into experimental conditions, often using an attribute value representation where features are a vector of gene expression values. 
This approach has the advantage that the constructed model may have found dependencies between genes which would not have been identified otherwise.
%(experimental conditions). 
Researchers will use the \lq top\rq\ features from the model to identify the set of genes to take on to stage 3. 
%Genes that are identified in this manner may be  different to a simple list of differently expressed genes as the constructed model , which are not highly differently expressed. 


%We note classification is often not the true task -the classification performance is used as a measure of the quality of the identified differences. Classification might not be the true task as the class of a sample may already be known - for example where the biopsy is taken from may indicate the class. 
%typical approach stage 2
In stage 3  researchers look for connections between these genes, for example by performing Gene Set Enrichment Analysis(GSEA) \citep{subramanian_gene_2005}. Here the set of identified genes are compared with predefined sets of genes. The predefined sets of genes indicate a known relation. For example having a related function, existing in the same location in the cell or taking part in the same pathway. A further way a set could be defined is a topological set, which takes into account finer grained internal relations to define a set. For example a topological set could be identified by community detection in a gene regulatory network.

%related relational work
To bring background knowledge of relations into the model building process, past ILP research \citep{gamberger_induction_2004} integrated stage 2: finding differently expressed genes and stage 3: GSEA. This was done by using Relational Subgroup Discovery. RSD has the advantage of being able to construct novel sets by sharing variables across predicates that define the sets. For example a set could be defined as the genes that have been annotated with two Gene Ontology terms.  Other ways researchers have tried to integrate the use of known relations is by adapting the classification approach. New features are built by aggregating across a predefined set of genes - for example by taking an average expression value for a pathway, see ~\citep{holec_comparative_2012} for a review of these methods.

In most of these methods it is common to ignore the detailed relations between entities in a pathway, the pathway is treated as an unstructured set of genes. Only do topologically defined sets take advantage of any known internal relations. On the other hand these often create crude clusters of genes, and do not follow biological intuition of information flow through the pathway. e.g. by using an inappropriate network abstraction or failing to adapt the clustering method to work on an appropriate network abstraction. When considering biological pathways simple directed networks of genes and proteins are not appropriate. This is because this formalism does not adequately model the dependencies of biochemical reactions i.e. bipartite graphs or hypergraphs are required. See \citep{whelan2011representation} for more details on this. 

A major limitation of the current classification approaches is that the models are constructed from either genes or crude aggregates of sets of genes. However biologists are not only interested in these kind of models. More general pathway activation patterns are also of interest.
% including paths and loops where some of the entities could possibly be represented by uninstantiated variables. 
Two examples are i). Consistency modelling and ii). Network motif finding.

In consistency modelling \citep{guziolowski2010analysis} a pattern of inconsistency is matched against gene regulatory networks and for each match Answer Set Programming techniques are used to amend the pathways to remove the inconsistency. Similar work has been carried out in ILP  where biological pathways have been constructed and amended \citep{ray2010automatic}. 

In network motif finding \citep{kim_biological_2011}, it is also common to ignore known biological concepts. Most motif finding algorithms are applied to a network representation which is inappropriate for pathways. There is also a failure to adapt the motif finding algorithm to take into account biological knowledge. In addition the patterns are often described in a language which is not as expressive as first order logic. This means they are unable to find patterns with uninstantiated variables, or to find patterns with relational concepts such as paths or loops. It is also the case that motifs are rarely identified for the purpose of discrimination in a classification task. Motifs are most commonly searched for across a single network constructed from an agglomerative of the experimental data, rather than by mapping expression values to a standard graph of the pathway. This approach would give an instantiated graph for each sample and enables the searching for pathway activation patterns (using the node states) across samples - taking into account the class of each sample. 

This work is most similar to  \citep{holec2008using} in which the authors propose using Fully Coupled Fluxes (FCF) as features. An FCF is the longest possible chain of vertices in which non-zero vertex activation implies a certain (non-zero) activation in its successors.

The aim of this paper is to identify pathway activation patterns that differ between biological samples of different classes, in order to give a biologist different information than models built from simple gene features. In contrast to the work on consistency modelling but consistent with the FCF approach, we assume the pathways are an accurate representation of biological processes and we are considering individual sample expression patterns mapped on to the pathway. 


%\section{Aim}
%

	  
\section{Method}
 We use structural pathway information to build first order features that are used to construct classification and ranking models, which  provide an appropriate level of abstraction for the interpretation of the model to be insightful for a biologist. 
\subsection{Raw Data} 
We obtained from the GEO a two class Lung Cancer data set containing 37 SCC examples and 33 AC examples; the accession number is GSE2109.The classes correspond to different types of Lung Cancer. For details of this task please see \citep{rhrissorrakrai_sbv_2013}.  
We use the Reactome database to provide the background knowledge about pathways. Reactome \citep{croft_reactome_2013} is a collection of manually curated peer reviewed pathways. Reactome is made available as an RDF XML Biopax level 3 file file, which allows for simple passing using SWI-Prolog's semantic web libraries. 


\subsection{Data Processing}
Reactome uses the bipartite network representation of entities and reactions- we extract this and process it to create a reaction centric graph, where nodes are reactions and directed edges are labelled either as \lq activation\rq\ ,\lq inhibition \rq\ or \lq follows\rq\ corresponding to how reactions are connected. 
Boolean networks~\citep{wang_boolean_2012} are a common abstraction in biological research, but these are normally applied at the gene or protein level not at the reaction level. In order to use a boolean network abstraction on a reaction network, we apply a logical aggregation method from the measured probes in the microarray to reactions.

We first discretize the probe values into binary values using Barcode \citep{mccall_gene_2014} which is a tool for applying learnt thresholds to microarray data,  Barcode makes it possible to compare gene expressions, both within a sample and between samples potentially measured by different arrays. 

Once we have binary probe values, we use the structure provided by
the Reactome RDF graph and our biological understanding to build reaction level features. Each reaction has a set of inputs and set of outputs, in addition a reaction may be controlled (activated or inhibited) by an entity. Entities in Reactome include protein complexes and protein sets, which in turn can themselves comprise of other complexes or sets. For the logical aggregation step we interpret this structure as a simple logical circuit. The relationship between probes and proteins is treated as an OR gate, protein complexes as an AND gate, and protein sets also as an OR gate. With a final step taking into account the activation or inhibition of the reaction by any controlling entities. In this way we can say that a reaction \lq can proceed \rq\ if and only if, the input circuit evaluates to true. If a reaction \lq can proceed' \rq\ then we say that it is \lq on\rq\ otherwise we say that it is \lq off \rq. %In this way we have created a biologically informed aggregate set of features at the reaction level, which has used the internal structure of a pathway in their construction. 

\subsection{Searching for Pathway Activation Patterns}
In this work we have experimented with two propositionalization methods, separately and then combining them. We search for features independently in each pathway and with a language bias that suitably constrains the search in order to achieve a manageable number of structures. 

The first method Warmr \citep{dehaspe_mining_1997} is the first order equivalent of item set and association rule mining.It can be used as propositionalization method by independently searching for frequent queries in the two classes. An advantage of Warmr is that it is possible to define background predicates to define relevant concepts. For example a path or loop of all \lq on\rq\ reactions. As Warmr does not prune by relevance to classification tasks it can however quickly build to an intractable search with many irrelevant or similar queries/features built. 
	     

The second method, TreeLiker~\citep{vzelezny2013fast} is a modern ILP tool that implements a number of algorithms. It has been shown to produce long features by building features bottom up in a blockwise manner. This is a desirable trait for  this task as longer features will provide more mechanistic insight to a biologist. A limitation is that the features are \lq tree like \rq\ which means there can be no cycles in the variables. TreeLiker unlike Warmr does not support explicit background knowledge and therefore all relevant relations need to be preprocessed using prolog.

For a combined method we first take a top feature constructed by TreeLiker and use this as the basis for the language bias input into Warmr. We then add further language bias constraints that guide Warmr into searching for frequent queries that create cycles in the tree like features. This results in longer features that contain cycles than Warmr can find on its own. 
	     
	     
\section{Results}
As we are interested in producing comprehensible models, we limit our experiments to rule (JRIP) and tree building algorithms (J48).We employ Wekas implementation. The best classifiers are in the Apoptosis pathway and achieve 81.29\% accuracy (std 13\%) in 10 fold cross validation.
\newpage


\noindent
{\it{Example features found by 1) Warmr, 2) TreeLiker and 3) Combined}}
\begin{verbatim}
1:array(A),reaction(A,B,1),reaction(A,C,0),link(C,B,D),link(B,C,E).
\end{verbatim}
\noindent 
This is a simple cyclical feature, the variable A matches one sample. 

\noindent

\begin{verbatim}
2:reaction(A,0), link(A,B,follows), reaction(B,1), link(B,C,_),    
reaction(C,0), link(A,D,activation), reaction(D,0).
\end{verbatim}
\noindent 
This is a longer tree like feature -notice TreeLiker does not require a variable for the sample.  

\noindent

\begin{verbatim}
3:array(A),reaction(A,B,0),link(B,C,follows),reaction(A,C,1),
link(C,D,E),reaction(A,D,0),link(B,F,activation),reaction(A,F,1),
link(F,D,E),link(D,G,E),reaction(A,G,0)
\end{verbatim}
\noindent
This final feature is both long and contains a cycle. With the structured features found, we were able to achieve a comparable accuracy to that of a model built with raw expression values - but now we have identified pathway activation perturbations rather than just gene expression perturbations. 

\section{Discussion}
The Pathway Activation Patterns that are found using this approach are in clinically relevant pathways. These patterns may give diagnostic and clinical insights that biologists can develop into new hypotheses for further investigation. 

This work has shown the appropriateness of using ILP methods to mine the abundance of highly structured biological data. Using this method we have identified differences in Pathway Activation Patterns that go beyond the standard analysis of differentially expressed genes, enrichment analysis, gene feature ranking and pattern mining for common network motifs. We have also demonstrated the use of logical aggregation to a  reaction graph and how this simplifies the search for hypothesises to an extent where searching all pathways is tractable. The extension of TreeLiker found features with Warmr to connect cycles is believed to be novel. This enables the finding of long cyclical features. 
\newline



%\subsubsection*{Acknowledgments.} Thanks to ...


\begingroup

\makeatletter
\patchcmd{\@makeschapterhead}{\vskip40}{\vskip12}{}{} 
\makeatother

\let\clearpage\relax
\bibliography{mybib}{}
\endgroup

%\bibliography{mybib}{}
%\bibliographystyle{plain}
\bibliographystyle{unsrt}
%\bibliographystyle{unsrtnat}
%\bibliographystyle{ieeetr}





\end{document}
